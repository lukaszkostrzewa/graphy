\chapter*{Wstęp}
\addcontentsline{toc}{chapter}{Wstęp}

\blockquote{This question is so banal, but seemed to me worthy of attention in that geometry, nor algebra, nor even the art of counting was sufficient to solve it.}

Tak pisał w liście Leonhard Euler\footnote{cytat za ...} o jednym z pierwszych problemów w~teorii grafów -- problemie mostów królewskich \cite[120]{einstein}. Banalny, ale~warty uwagi. W dzisiejszych czasach teoria grafów rozwiązuje wiele nietrywialnych problemów, a część z nich nadal pozostaje otwarta\footnote{np. hipoteza Hadwigera, Chvatala, Vizinga}. Grafy znalazły praktyczne zastosowanie w wielu różnorodnych dziedzinach nauki, takich jak informatyka, ekonomia, socjologia, jak również chemia, lingwistyka, geografia czy nawet architektura. Bez wątpienia teoria grafów jest dziedziną matematyki i informatyki, która zasługuje na uwagę, co postaram się w niniejszej pracy przedstawić.

Głównym celem mojej pracy jest stworzenie aplikacji służącej do wizualizacji i edycji grafów w przeglądarce. W przeciągu kilku ostatnich lat mogliśmy zaobserwować gwałtowny wzrost znaczenia aplikacji internetowych. Co~dziwne, na~dzień dzisiejszy w sieci praktycznie nie ma rozwiązania, które pozwalałoby wczytać graf, wyświetlić, w łatwy sposób przetworzyć, a następnie wyeksportować do~znanego formatu. Praca ta jest odpowiedzią na~ów~deficyt. 

W pracy dokonam również przeglądu i analizy bibliotek JavaScript oraz technologii służących do wizualizacji grafów w przeglądarce.

\thispagestyle{empty}
